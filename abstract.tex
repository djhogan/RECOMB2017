\documentclass{article}
\usepackage[margin=1.0in]{geometry}

\begin{document}
\noindent Title: SeqMiner: A Weka package for mining phage display sequence data. \\
Authors: \\
\indent Daniel J. Hogan [djh901@mail.usask.ca] (1), \\
\indent Bharathi Vellalore (2), \\
\indent Clarence R. Geyer (2), and \\
\indent Anthony J. Kusalik (1) \\
Affiliations: \\
\indent(1) University of Saskatchewan, Dept. of Computer Science; \\
\indent(2) University of Saskatchewan, Dept. of Biochemistry \\

Many standard classification and regression models used in data mining are able to handle numeric and nominal attributes, but few are able to handle string attributes like protein sequences. We have created a Weka package called SeqMiner for extracting numeric features from protein sequences that can be used to train and test classifiers in the Weka machine learning workbench. SeqMiner contains classes for extracting three sets of numeric features from protein sequences: (1) amino acid counts, (2) dipeptide counts, (3) and Kidera factors (i.e. 10 principal components from a PCA of hundreds of physicochemical properties). 

A Weka KnowledgeFlow using SeqMiner classes was used to extract features from the CDR-H3 sequences of several antibody phage display (APD) experiments. The features included (1) amino acid counts, (2) dipeptide counts, and (3) Kidera factors. The features were used to train and test a random forest classifier to predict CDR-H3 sequences that become enriched during the APD experiments, which indicates binding to the target. The random forest achieved a classification accuracy of 84.4\%. One potential application of this classifier is in the design of APD libraries with sequence landscapes yielding more binders.

SeqMiner can drastically reduce the effort required for data mining of short protein sequences like those generated by phage display. As part of the Weka ecosystem, SeqMiner benefits from Weka's many features, including KnowledgeFlow. The SeqMiner package is available on github (github.com/djh901/seqminer). Feature requests may be sent to djh901@mail.usask.ca.

% A second KnowledgeFlow was created to identify features from feature set 4 that correlated the best with the enrichment class variable. The top 10 variables in order from best to worst were Kidera 6, Kidera 1, Y, A, L, C, DY, R, and S.
\end{document}
